\documentclass[uplatex,dvipdfmx]{jsarticle}

\usepackage[uplatex,deluxe]{otf} % UTF
\usepackage[noalphabet]{pxchfon} % must be after otf package
\usepackage{stix2} %欧文&数式フォント
\usepackage[fleqn,tbtags]{mathtools} % 数式関連 (w/ amsmath)
\usepackage{hira-stix} % ヒラギノフォント&STIX2 フォント代替定義(Warning回避)

\begin{document}

\title{BBS新機能追加仕様書}
\author{24G1105 長妻冬威}
\date{2024年 1月7日}

\begin{document}

\maketitle


\section{利用者向けの仕様書}
\subsection{システム概要}
本bbsでは、以下の新機能をBBSに追加した.
\begin{itemize}
    \item 投稿への「いいね」機能
    \item 投稿編集機能
    \item 投稿削除機能
    \item スレッド検索機能
\end{itemize}

\subsection{利用手順}
各機能の使い方は以下のように行う.
\begin{enumerate}
    \item いいね:投稿一覧からIDを確認し,いいねボタンを押して対象の投稿IDを入力する.
    \item 編集:編集したい投稿のIDを入力し,新しいメッセージを入力する.
    \item 削除:削除したい投稿のIDを入力する.
    \item 検索:検索フォームにキーワードを入力し,検索結果を表示する.
\end{enumerate}

\section{管理者向け仕様書}
\subsection{システム構成}
システムは以下の技術で構築されている.
\begin{itemize}
    \item サーバ:node.js + Express
    \item クライアント:HTML, JavaScript
    \item データ形式:JSON
\end{itemize}

\subsection{サーバの設定と動作}
サーバは以下のコマンドで起動する.
\begin{verbatim}
$ node app7.js
\end{verbatim}
このコマンドを入力し,別のターミナルを開き以下のコマンドを入力する.
\begin{verbatim}
    $ telnet localhost 8080
\end{verbatim}
サーバは\texttt{http://localhost:8080/public/bbs.html}を検索することで使用することができる.

\subsection{データ構造}
投稿データは以下の形式で管理される.
\begin{verbatim}

    id:0,
    name:"投稿者名",
    message:"メッセージ内容",
    likes:0

\end{verbatim}

\section{開発者向け仕様書}
\subsection{API仕様}
\subsubsection{/post}
新しい投稿を作成する.
\begin{itemize}
    \item メソッド: POST
    \item リクエスト例:
    \begin{verbatim}
    { name: "Alice", message: "こんにちは" }
    \end{verbatim}
    \item レスポンス例:
    \begin{verbatim}
    { number: 1 }
    \end{verbatim}
\end{itemize}

\subsubsection{/like}
投稿にいいねを追加する.
\begin{itemize}
    \item メソッド: POST
    \item リクエスト例:
    \begin{verbatim}
    { id: 0 }
    \end{verbatim}
    \item レスポンス例:
    \begin{verbatim}
    { likes: 1 }
    \end{verbatim}
\end{itemize}


\section{ソースコードと実行例}
ソースコードは以下のURLに示してある.
\begin{itemize}
    \item \url{https://github.com/Nagatsumatoui/webpro_06}
\end{itemize}


\end{document}
